\documentclass{article}
\title{PCA Functions User Manual}
\author{Chen Chun-Yu}
\textwidth=6.2in
\textheight=8.5in
\oddsidemargin=.1in
\evensidemargin=.1in
\headheight=-.3in
\usepackage{subfigure}

\usepackage{Sweave}
\begin{document}
\Sconcordance{concordance:pca_function_manual.tex:pca_function_manual.Rnw:%
1 10 1 1 0 136 1}

\maketitle
\tableofcontents
\newpage
\section{Introduction}
This manual shows you how to use the PCA function to perform the analysis and gives an example for demonstration.
\section{Prerequest}
To make the functions run successfully, please ensure you have installed PLINK in previous.\\
After you installed PLINK, please copy it from defult folder and paste it to /usr/local/bin.
\\\\\verb@$ cp ~/Bin/plink /usr/local/bin@\\\\\
PCA function will detect whether you have installed three required packages: "gdsfmt", "SNPRelate", and "randomcoloR", 
if you short of any of these it will install for you automatically.
\section{PCA Functions}
\begin{itemize}
    \item my\_pca(plink.file, data.clust, pop.list, output.name)
    \begin{itemize}
    \item plink.file \\
    Please provide the name of your PLINK binary files(BED/BIM/FAM).
    \end{itemize}
    \begin{itemize}
    \item data.clust \\
    Please provide a list of indviduals with their population names which correspond to your FAM file.
    An example file format as below:
    \begin{table}[ht]
    \centering
    \begin{tabular}{lcr}
    \hline
    Family ID & Individual ID & Populations \\ 
    \hline
    1 & TDC13 & Paiwan \\ 
    2 & TDC117 & Amis \\ 
    3 & TDC18 & Bunun \\ 
    4 & TDC129 & Amis \\
    5 & TDC49 & Amis \\
    6 & TDC497 & Puyuma \\
    \hline
    \end{tabular}
    \end{table}
    \end{itemize}
    \begin{itemize}
    \item pop.list \\
    Please provide a list of populations that you are going to perform PCA.
    An example file format as below:
    \begin{table}[ht]
    \centering
    \begin{tabular}{lcr}
    \hline
    Populations \\ 
    \hline
    Paiwan \\ 
    Amis \\ 
    Bunun \\ 
    Puyuma \\
    \hline
    \end{tabular}
    \end{table}
    \end{itemize}
    \begin{itemize}
    \item output.name \\
    You need to name your output files. After the analysis, you will get three plots: PCA outcome("output.name\_pca.pdf"), 
    pair plot for the first four PCs("output.name\_pairs.pdf"), and line plot that shows different variations capture by 
    different components("output.name\_var.pdf").
    Besides, you will also get a table which shows the detail information of PCA, including values of PC1 and PC2.("output.name\_pca.tab")
    \end{itemize}
\end{itemize}
\section{Examples}
Here we give you some examples of applying PCA functions we have illustrated. We used three different sets
of populations for analysis: Taiwanese aborigines, Han populations, and Philipine populations.
\begin{itemize}
    \item Taiwanese aborigines
    \begin{figure}[h]
    \centering
    \subfigure[PCA of Taiwanese aborigines]
    {
        \includegraphics[width=2.8in]{/Users/chenchun-yu/Project/raw_data/aborigines_pop_pca.pdf}
    }
    \\
    \subfigure[Pair plot for the first four PCs.]
    {
        \includegraphics[width=2.8in]{/Users/chenchun-yu/Project/raw_data/aborigines_pop_pairs.pdf}
    }
    \subfigure[Variation captured by different PCs.]
    {
        \includegraphics[width=2.8in]{/Users/chenchun-yu/Project/raw_data/aborigines_pop_var.pdf}
    }
    \caption{Example 1 - Three kinds of output plots.}
    \end{figure}
\end{itemize}
\newpage
\begin{itemize}
    \item Han populations
    \begin{figure}[h]
    \centering
    \subfigure[PCA of Han populations]
    {
        \includegraphics[width=2.8in]{/Users/chenchun-yu/Project/raw_data/han_pop_pca.pdf}
    }
    \\
    \subfigure[Pair plot for the first four PCs.]
    {
        \includegraphics[width=2.8in]{/Users/chenchun-yu/Project/raw_data/han_pop_pairs.pdf}
    }
    \subfigure[Variation captured by different PCs.]
    {
        \includegraphics[width=2.8in]{/Users/chenchun-yu/Project/raw_data/han_pop_var.pdf}
    }
    \caption{Example 2 - Three kinds of output plots.}
    \end{figure}
\end{itemize}
\newpage
\begin{itemize}
    \item Philipines populations
    \begin{figure}[h]
    \centering
    \subfigure[PCA of Philipines populations]
    {
        \includegraphics[width=2.8in]{/Users/chenchun-yu/Project/raw_data/philipine_pop_pca.pdf}
    }
    \\
    \subfigure[Pair plot for the first four PCs.]
    {
        \includegraphics[width=2.8in]{/Users/chenchun-yu/Project/raw_data/philipine_pop_pairs.pdf}
    }
    \subfigure[Variation captured by different PCs.]
    {
        \includegraphics[width=2.8in]{/Users/chenchun-yu/Project/raw_data/philipine_pop_var.pdf}
    }
    \caption{Example 3 - Three kinds of output plots.}
    \end{figure}
\end{itemize}
\begin{thebibliography}{1}
\bibitem{notes} Purcell S, Neale B, Todd-Brown K, et al. {PLINK: A Tool Set for Whole-Genome Association and Population-Based Linkage Analyses. 
\em American Journal of Human Genetics.} 2007;81(3):559-575.
\bibitem{notes} Zheng X, Levine D, Shen J, Gogarten SM, Laurie C, Weir BS. {A high-performance computing toolset for relatedness and principal component analysis of SNP data. \em Bioinformatics.} 2012;28(24):3326-3328. doi:10.1093/bioinformatics/bts606.
\end{thebibliography}
\end{document}
